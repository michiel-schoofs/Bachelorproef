%%=============================================================================
%% Samenvatting
%%=============================================================================



%%---------- Nederlandse samenvatting -----------------------------------------
%
% TODO: Als je je bachelorproef in het Engels schrijft, moet je eerst een
% Nederlandse samenvatting invoegen. Haal daarvoor onderstaande code uit
% commentaar.
% Wie zijn bachelorproef in het Nederlands schrijft, kan dit negeren, de inhoud
% wordt niet in het document ingevoegd.

\IfLanguageName{english}{
\selectlanguage{dutch}
\chapter*{Samenvatting}

\selectlanguage{english}
}{}

%%---------- Samenvatting -----------------------------------------------------
% De samenvatting in de hoofdtaal van het document

\chapter*{\IfLanguageName{dutch}{Samenvatting}{Abstract}}
% TODO: De "abstract" of samenvatting is een kernachtige (~ 1 blz. voor een
% thesis) synthese van het document.
%
% Deze aspecten moeten zeker aan bod komen:
% - Context: waarom is dit werk belangrijk? OK
% - Nood: waarom moest dit onderzocht worden? OK
% - Taak: wat heb je precies gedaan? OK
% - Object: wat staat in dit document geschreven? OK
% - Resultaat: wat was het resultaat? OK
% - Conclusie: wat is/zijn de belangrijkste conclusie(s)? OK
% - Perspectief: blijven er nog vragen open die in de toekomst nog kunnen
%    onderzocht worden? Wat is een mogelijk vervolg voor jouw onderzoek?
%
% LET OP! Een samenvatting is GEEN voorwoord!
Software is onderworpen aan constante veranderingen. Zo moeten er nieuwe functies voorzien worden en fouten opgelost. Dit is zeker moeilijk als er verschillende ontwikkelaars moeten samenwerken aan één project. Om dit te vergemakkelijken steunen softwareontwikkelaars op versiebeheersystemen. Deze systemen hebben echter één groot nadeel. Ze vereisen een centrale server om bestanden en versies te gaan opslaan.  Dit is problematisch doordat een server offline of beschadigd kan worden. Hierdoor kunnen hele projecten verloren gaan. Deze bachelorproef stelt aldus een werkwijze voor om versiebeheer te decentraliseren. Dit houdt in dat er geen centrale server meer is om de bestanden en versies te gaan opslaan. In plaats van een centrale server wordt er geopteerd voor een netwerk van computers. Daardoor staan de verschillende versies verspreidt over heel het netwerk en is er geen centrale plaats waar het fout kan gaan.\\

Om deze oplossing mogelijk te maken wordt er gebruik gemaakt van blockchain technologie en IPFS - een gedecentraliseerd file-sharing protocol-. Deze begrippen worden eerst gekaderd in een literatuurstudie. Vervolgens wordt aan de hand van een prototype een werkbaar gedecentraliseerd versiebeheer systeem geïmplementeerd. Dit prototype houd daarbij rekening met enkele succescriteria\\

De oplossing die ontwikkeld werd in deze bachelorproef voldoet aan alle functionele vereisten van een versiebeheersysteem. Deze bachelorproef is aldus geslaagd in haar opzet. Toch is het geen gangbaar alternatief voor populaire versiebeheersystemen zoals GitHub. Dit komt doordat de gebruikte oplossing gebouwd is bovenop het Ethereum netwerk. Dit netwerk vereist transactiekosten voor elke handeling die data wegschrijft op de blockchain. Het prototype dat ontwikkeld werd is aldus niet kost effectief. Zeker als men in beschouwing neemt dat meeste versiebeheersystemen gratis zijn lijkt het niet plausibel dat mensen zouden overstappen op deze oplossing.\\

Er zijn echter ook netwerken die geen transactiekosten vereisen waaronder het Loom netwerk. Indien deze oplossing verder zou ontwikkeld worden kan er dus gesteund worden op dit type van netwerken.