%%=============================================================================
%% Inleiding
%%=============================================================================

\chapter{\IfLanguageName{dutch}{Inleiding}{Introduction}}
\label{ch:inleiding}
\textbf{Moet nog herwerkt worden. Verschillende zaken moeten nog aangevuld.}

Software ontwikkeling is een proces dat constant evolueert. Ze komen er voortdurend nieuwe oplossingen op de markt. Toch zijn er enkele fundamentele zaken die relevant blijven. Zo wordt er gedurende het software ontwikkelingsproces gesteund op het concept van versiebeheersystemen om code te gaan beheren. Deze systemen bieden een manier om wijzigingen in code te gaan bijhouden en onder te verdelen in versies. Stellen software ontwikkelaars ook instaat om op een eenvoudige wijze samen te gaan werken aan grote software projecten\\

Toch zijn deze systemen in handen van verschillende grote tech-giganten waaronder Microsoft. Dit is eerder problematisch als men realiseert dat veel open-source bedrijven steunen op deze oplossingen. Er is tevens nog geen populair alternatief voor handen om versiebeheer democratisch en zonder tussenkomst van derden te laten verlopen.\\

Toch is het door middel van nieuwe technologie zoals Blockchain en IPFS perfect mogelijk om gangbaar alternatief te ontwikkelen zonder dat daar een derde partij bij hoeft betrokken te worden.


\section{\IfLanguageName{dutch}{Onderzoeksvraag}{Research question}}
\label{sec:onderzoeksvraag}
De onderzoeksvraag luidt dus als volgt: \textbf{Hoe kan men doormiddel van Blockchain principes en IPFS een volledig en transparant versiebeheersysteem bekomen?}


\section{\IfLanguageName{dutch}{Onderzoeksdoelstelling}{Research objective}}
\label{sec:onderzoeksdoelstelling}

De doelstelling van deze bachelorproef is om een werkend proto-type te ontwerpen dat voldoet aan onderstaande criteria:

\begin{enumerate}
	\item De oplossing die wordt ontwikkeld mag geen centrale component hebben en moet op een volledig gedistribueerde manier in staat zijn om te functioneren.
	\item De oplossing moet een gebruiker instaat stellen om documenten met anderen te kunnen delen.
	\item De oplossing moet een manier aanbieden om wijzigingen aan deze documenten op te slaan onder de vorm van verschillende versies. Deze verschillende versies moeten kunnen geraadpleegd worden.
\end{enumerate}

Wat is het beoogde resultaat van je bachelorproef? Wat zijn de criteria voor succes? Beschrijf die zo concreet mogelijk. Gaat het bv. om een proof-of-concept, een prototype, een verslag met aanbevelingen, een vergelijkende studie, enz.

\section{\IfLanguageName{dutch}{Opzet van deze bachelorproef}{Structure of this bachelor thesis}}
\label{sec:opzet-bachelorproef}

% Het is gebruikelijk aan het einde van de inleiding een overzicht te
% geven van de opbouw van de rest van de tekst. Deze sectie bevat al een aanzet
% die je kan aanvullen/aanpassen in functie van je eigen tekst.

De rest van deze bachelorproef is als volgt opgebouwd:

In Hoofdstuk~\ref{ch:stand-van-zaken} wordt een overzicht gegeven van de stand van zaken binnen het onderzoeksdomein, op basis van een literatuurstudie.

In Hoofdstuk~\ref{ch:methodologie} wordt de methodologie toegelicht en worden de gebruikte onderzoekstechnieken besproken om een antwoord te kunnen formuleren op de onderzoeksvragen.

% TODO: Vul hier aan voor je eigen hoofstukken, één of twee zinnen per hoofdstuk

In Hoofdstuk~\ref{ch:conclusie}, tenslotte, wordt de conclusie gegeven en een antwoord geformuleerd op de onderzoeksvragen. Daarbij wordt ook een aanzet gegeven voor toekomstig onderzoek binnen dit domein.