%%=============================================================================
%% Inleiding
%%=============================================================================

\chapter{\IfLanguageName{dutch}{Inleiding}{Introduction}}
\label{ch:inleiding}
\subsection{Introductie versiebeheersystemen}
Elke software ontwikkelaar kan beamen dat software fel onderhevig is aan veranderingen. Constant moeten er nieuwe functies worden geïmplementeerd en fouten worden rechtgezet. Dit probleem wordt enkel verergerd indien er tientallen mensen samen aan deze projecten werken. Versiebeheersystemen bieden een oplossing voor deze problemen. Tegenwoordig zijn ze een onmisbaar deel geworden van de manier waarop software wordt ontwikkeld.\\

Deze systemen delen software op in een aaneenschakeling van veranderingen. Deze veranderingen worden ook wel versies genoemd. Deze systemen maken het mogelijk om nieuwe wijzigingen aan te brengen en toegang te krijgen tot oudere versies. Moderne versiebeheersystemen bieden daarbovenop ook de mogelijkheid om twee conflicterende wijzigingen samen te voegen. Hierdoor wordt samenwerken aan grote projecten vergemakkelijkt.\\

Kort samengevat bestaat een versiebeheersysteem uit twee verschillende functies:

\begin{enumerate}
\item Een plaats waar bestanden kunnen worden opgeslagen en opgevraagd worden.
\item Een manier om wijzigingen aan te brengen en toegang te krijgen tot eerdere versies van deze bestanden.
\end{enumerate}

Versiebeheersystemen stellen softwareontwikkelaars instaat om code publiek toegankelijk te maken. Hierdoor kan elke software ontwikkelaar de broncode gaan raadplegen en aanpassen. Een principe dat mooi kadert binnen de zogenaamde Open-Source filosofie. Deze filosofie streeft naar een open en democratische ontwikkeling van software. Hierbij staat samenwerken aan verbeteringen centraal. Software projecten worden aldus aanzien als een gezamenlijk project.

\subsection{Probleem stelling}
Toch is er een probleem met de manier waarop deze systemen functioneren. Ze vereisen namelijk een centrale server voor de bestanden en wijzigingen te gaan opslaan. Dit is problematisch aangezien deze servers eventueel beschadigd of onbereikbaar kunnen zijn. Hierdoor kunnen projecten onbeschikbaar zijn of in het ergste geval zelfs verloren gaan. Dit probleem staat ook wel bekend onder de naam \textbf{Single point of failure}.\\

Een ander probleem is dat deze servers in de handen kunnen komen van grote tech-giganten. Zo heeft Microsoft in 2018 GitHub -een populair versiebeheer platform- overgekocht. Dit druist in tegen de eerder vermelde Open-Source filosofie. Want deze systemen zijn niet gevrijwaard van commerciële invloeden.\\

\subsection{Introductie decentralisatie}

Toch is het door middel van nieuwe technologie zoals Blockchain en IPFS perfect mogelijk om gangbaar alternatief te ontwikkelen zonder dat daar een derde partij bij hoeft betrokken te worden.


\section{\IfLanguageName{dutch}{Onderzoeksvraag}{Research question}}
\label{sec:onderzoeksvraag}
De onderzoeksvraag luidt dus als volgt: \textbf{Hoe kan men doormiddel van Blockchain principes en IPFS een volledig en transparant versiebeheersysteem bekomen?}


\section{\IfLanguageName{dutch}{Onderzoeksdoelstelling}{Research objective}}
\label{sec:onderzoeksdoelstelling}

De doelstelling van deze bachelorproef is om een werkend proto-type te ontwerpen dat voldoet aan onderstaande criteria:

\begin{enumerate}
	\item De oplossing die wordt ontwikkeld mag geen centrale component hebben en moet op een volledig gedistribueerde manier in staat zijn om te functioneren.
	\item De oplossing moet een gebruiker instaat stellen om documenten met anderen te kunnen delen.
	\item De oplossing moet een manier aanbieden om wijzigingen aan deze documenten op te slaan onder de vorm van verschillende versies. Deze verschillende versies moeten kunnen geraadpleegd worden.
\end{enumerate}

Wat is het beoogde resultaat van je bachelorproef? Wat zijn de criteria voor succes? Beschrijf die zo concreet mogelijk. Gaat het bv. om een proof-of-concept, een prototype, een verslag met aanbevelingen, een vergelijkende studie, enz.

\section{\IfLanguageName{dutch}{Opzet van deze bachelorproef}{Structure of this bachelor thesis}}
\label{sec:opzet-bachelorproef}

% Het is gebruikelijk aan het einde van de inleiding een overzicht te
% geven van de opbouw van de rest van de tekst. Deze sectie bevat al een aanzet
% die je kan aanvullen/aanpassen in functie van je eigen tekst.

De rest van deze bachelorproef is als volgt opgebouwd:

In Hoofdstuk~\ref{ch:stand-van-zaken} wordt een overzicht gegeven van de stand van zaken binnen het onderzoeksdomein, op basis van een literatuurstudie.

In Hoofdstuk~\ref{ch:methodologie} wordt de methodologie toegelicht en worden de gebruikte onderzoekstechnieken besproken om een antwoord te kunnen formuleren op de onderzoeksvragen.

% TODO: Vul hier aan voor je eigen hoofstukken, één of twee zinnen per hoofdstuk

In Hoofdstuk~\ref{ch:conclusie}, tenslotte, wordt de conclusie gegeven en een antwoord geformuleerd op de onderzoeksvragen. Daarbij wordt ook een aanzet gegeven voor toekomstig onderzoek binnen dit domein.