\setcounter{section}{0}
\section{Basis syntax van de taal}
\label{ch:Voorbeelden van solidity contracten} 
Onderstaand contracten tonen enkele van de concepten en werkwijzen in Solidity. De onderstaande code voorbeelden zijn voorzien van enkele commentaar regels ter aanvulling. Aangezien de focus in deze Bachelorproef eerder ligt op de implementatie wordt er niet uitgebreid ingegaan op Solidity. Deze voorbeeld contracten dienen dus enkel als een ondersteuning bij codevoorbeelden. Voor meer informatie raadpleeg de officiële documentatie \cite{SolidityDocs}. Tot slot de verschillende contracten zijn getest op de versies van Solidity zoals gedefinieerd in de \textbf{pragma} tags.\\

\begin{lstlisting}[language=Solidity]
//specify the version of solidity to use
pragma solidity >=0.4.24;

contract Concepts {
    //A getter is automatically created with a public state variable
    string public name;
    //The way we define a constant value, can not be changed
    bool public constant alwaysTrue=true;
    //Only make this visible to our contract, people can't view the owner but we can reference it in our contract
    address internal owner;

    //Epoch time
    uint internal openingtime;

    //Modifiers are a way of limiting when a function can execute
    //Note overloaded Modifiers is not a thing that exists currently in solidity
    modifier onlyOwner(){
        //If the expression evaluates to false then throw exception
        //msg is metadata that gets added to every call, so we verify that the person calling this function is the owner
        require(msg.sender == owner,"Only the owner of this contract can run this command");
        //If normal execution don't do anything
        _;
    }

    //Allow the owner to change control
    //Note: You cannot add optional parameters however there is PR to add it.
    function changeOwner(address _newOwner) public onlyOwner {
        owner = _newOwner;
    }

    //Make a way that our owner can change the opening time
    function setOpeningtime(uint _openingTime) public onlyOwner {
        openingtime = _openingTime;
    }

    //The view modifier makes it clear we only intend this function to read data and not actually modify the state variables
    function getOpeningtime() public view returns (uint) {
        return openingtime;
    }

    //Enumerations also exist
    enum ContractStatus{
        Open, Closed
    }

    //We can use enumerations in much the same way
    function getStatus() public view returns (ContractStatus){
        if(block.timestamp >= openingtime) {
            return ContractStatus.Open;
        }

        return ContractStatus.Closed;
    }

    modifier ContractIsOpen() {
        //We can evaluate enumerations in this way
        require(getStatus() == ContractStatus.Open,"the contract is not opened");
        _;
    }

    // A way to represent custom datastructors
    struct People{
        string _firstname;
        string _lastname;
        uint8 _age;
    }

    //Comparable to a dictonary so a key value store
    mapping(uint8 => People) PeopleAr;
    //There's no way to querry a mapping for the keys or count so we have to manually add those variables
    uint8[] public ids = new uint8[](0);
    uint8 public size = 0;

    //This is how we set a value in our mapping and also make an instance of a struct
    function addPeople(string memory _fn, string memory _ln, uint8 _age) public returns (uint8) {
        //There's no lists but the concept of dynamic arrays function in much the same way
        ids.push(size);
        //Set a new person object with the id == size and instantiate a new Person object
        PeopleAr[size] = People(_fn,_ln,_age);
        //Short hand operator of size = size+1
        size++;
    }

    //Gets called when the contract is first deployed
    //Prevent using times because of an exploit
    constructor(string memory _name) public {
        //set state variable
        name = _name;
        owner = msg.sender;
        openingtime = now;
    }

    //Used to demo that function overloading is a thing that exists
    function functionToOverload(uint8 blabla) public pure returns(uint8) {
        return blabla;
    }

    //Parameters just need to be different types, uint8 and uint256 is in fact another type cool huh :)
    function functionToOverload(uint256 blabla) public pure returns(uint256) {
        return blabla;
    }
}
\end{lstlisting}
\section{Overerving}
\begin{lstlisting}[language=Solidity]
pragma solidity >=0.5.1;

//Contract to show how inheritance works
contract ERC20Token{
    string public name;
    mapping(address=> uint256) balance;

    constructor(string memory _name) public{
        name = _name;
    }

    //Make sure we can override the method with the virtual modifier
    function mint() public virtual {
        //The sender is a contract so we can't use msg.sender
        //We can use tx.origin to get the person initiaing a transaction
        balance[msg.sender]++;
    }
}


contract InheritanceContract  is ERC20Token {
    //You can't override state variabels
    //string override name = "MyToken";
    //You can have your own state variables inside of subclass
    string public symbol;
    address[] public owners;
    uint256 public ownerCount;

    //Super constructor call
    constructor(string memory _name, string memory _symbol) public ERC20Token(_name) {
        //Set property from super class
        symbol = _symbol;
    }

    //override function mint
    function mint() public override {
        super.mint();
        ownerCount++;
        owners.push(msg.sender);
    }
}
\end{lstlisting}
\section{Library voorbeeld}
\begin{lstlisting}[language=Solidity]
pragma solidity >=0.5.0;

//Library
//Promote code reuse
library Math {
    //A library just stores a set of functions that we can reuse so we only have to maintain the code in one place
    //A pure function basically just tells we're only using the variables passed into the function and no state variables
    //Good practice to make this public
    function devide(uint _val1, uint _val2) public pure returns (uint){
        require(_val2 > 0,"");
        return _val1/_val2;
    }
}
\end{lstlisting}
\begin{lstlisting}[language=Solidity]
contract UsingMath {
    uint256 public value;

    function calculate(uint _val1, uint _val2) public pure returns (uint) {
        //call the library
        return Math.devide(_val1,_val2);
    }
}
\end{lstlisting}
\section{geavanceerde concepten}
\begin{lstlisting}[language=Solidity]
pragma solidity >=0.5.1;

//Contract to show how we can talk to another contract.
contract ERC20Token{
    string public name;
    mapping(address=> uint256) balance;

    function mint() public {
        //The sender is a contract so we can't use msg.sender
        //We can use tx.origin to get the person initiaing a transaction
        balance[tx.origin]++;
    }
}


contract AdvancedConcepts {
    //Payable makes sure that the wallet is valid and can recieve ether
    address payable wallet;
    address public tokenContract;

    //Build in subscriber pattern, indexed fields can be filtered on when subscribing
    event SomeonePayed(
        address indexed _buyer,
        uint amount
    );

    constructor(address payable _owner, address _token) public {
        wallet = _owner;
        tokenContract = _token;
    }

    //default function that gets executed when calling this command also reffered to as a fallback function
    //When the fallback function is payable then use the recieve keyword if not you can use the fallback keyword
    receive() external payable {
        sendEther();
    }

    //A way to send ether to the wallet, payable makes sure we can actually attach ether to this function
    function sendEther() public payable {
        //transfer the atached amount of ehter to the wallet
        wallet.transfer(msg.value);
        //Gets pushed on the log, transactions are async and we can use this to listen effictively
        // We can also use this to know when a contract function got executed
        emit SomeonePayed(msg.sender, msg.value);
        //Get a reference to the contract and tell what it is
        ERC20Token _token = ERC20Token(address(tokenContract));
        //Call a function
        _token.mint();
    }
}
\end{lstlisting}
\section{Nieuwe concepten vanaf 0.6.0}
