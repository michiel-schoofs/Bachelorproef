%%=============================================================================
%% Conclusie
%%=============================================================================

\chapter{Conclusie}
\label{ch:conclusie}
De hoofdonderzoeksvraag die deze bachelorproef behandelde luide als volgt: 

\textbf{Hoe kan men door middel van Blockchain principes en IPFS een werkbaar gedecentraliseerd versiebeheersysteem ontwikkelen?}

Om deze vraag te beantwoorden moet men gaan definiëren wat een werkbaar gedecentraliseerd versiebeheersysteem is. Dit werd steunende op de literatuurstudie afgebakend door middel van concrete succesfactoren. Een overzicht van deze factoren vindt men in de tabel 3.1: Succescriteria van de POC. Vertrekkende vanuit deze factoren werd vervolgens een prototype ontwikkeld door middel van IPFS en blockchain. Het ontwikkelde prototype voldeed daarbij aan elk van deze factoren.\\ 

De conclusie luidt dus als volgt: \textbf{door de werkwijze te volgen zoals uitgelijnd in de methodologie -zie \ref{ch:methodologie}- kan men een werkbaar versiebeheersysteem bekomen gebruikmakend van IPFS en blockchain principes.}\\

Zoals ook vermeld in de inleiding was het de bedoeling dat deze oplossing een gangbaar alternatief bood voor open source bedrijven. In dat aspect is deze bachelorproef echter niet geslaagd. Veel van de huidige oplossingen zoals GitHub zijn namelijk gratis of vereisen een vaste kost. Doordat het prototype gebruik maakt van de Ethereum blockchain spelen transactiekosten een rol. Elke handeling die data gaat schrijven op onze blockchain zoals het opslaan van nieuwe bestanden vereist aldus een betaling. Daarom lijkt deze oplossing niet commercieel haalbaar.\\

\section{Verder onderzoek}
Er zijn veel concepten en vragen die niet realistisch waren om in de gegeven tijdspanne te behandelen. Er zijn dan ook enkele interessante vervolgstudies die uit dit onderzoek kunnen voortvloeien:

\begin{enumerate}
\item Het Ethereum netwerk vereist transactie kosten om data te gaan aanpassen met behulp van smartcontracts. Er zijn ook netwerken die geen transactie kosten vereisen voor gebruikers van gedecentraliseerde applicaties waaronder Loom. Zijn deze netwerken een gangbaar alternatief voor Ethereum?
\item Smartcontracts hebben een sterke focus op zo klein,veilig en efficiënt mogelijk te zijn. Daarom wordt er sterk gesteund op reeds bestaande implementaties zoals Open Zepplin. Is er een manier om zoals in klassiek programmeren verschillende Ontwerp Patronen te formaliseren die als leid draad kunnen dienen voor de ontwikkeling van smartcontracts?
\end{enumerate}

