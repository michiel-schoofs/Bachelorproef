%---------- Inleiding ---------------------------------------------------------

\section{Introductie} % The \section*{} command stops section numbering
\label{sec:introductie}

De onderzoeksvraag waaruit wordt vertrokken luidt als volgt: \textbf{"Hoe kunnen we doormiddel van IPFS en Blockchain technologie versiebeheer van een Client-Server architectuur naar een gedecentraliseerde Peer-to-peer model overzetten?"} Binnen deze onderzoeksvraag zijn er vier begrippen:\\\\
\begin{itemize}
\item \textbf{Versiebeheer}: Git -een grote speler op het gebied van Versiebeheer- hanteert volgende definitie van het begrip “Versiebeheer is het systeem waarin veranderingen in een bestand of groep van bestanden over de tijd wordt bijgehouden, zodat je later specifieke versies kan opvragen.” Deze definitie werd gepubliceerd in het boek \textit{Pro git} \autocite{Chacon2014}. \\\\

\item \textbf{IPFS}: Interplanetary File System of (IPFS)werd in de paper “IPFS - Content Addressed, Versioned, P2P File System” geïntroduceerd door \textcite{Benet2014}. Hierin stelde hij zijn technologie voor als een peer-to-peer gedistribueerd bestandssysteem waarin alle computers werken met dezelfde bestandsindeling. Hiermee wordt bedoeld dat bestanden kunnen worden opgedeeld in verschillende delen (\textit{ook wel 'Shards' genoemd}) en vervolgens worden opgeslagen op verschillende computers op een gezamenlijk netwerk. Vervolgens kunnen bestanden worden opgevraagd door middel van dit gezamenlijk netwerk aan te spreken. Er is dus geen gecentraliseerd aanspreekpunt.\\\\

\item \textbf{Blockchain}: Blockchain is een manier om data op te slaan aan de hand van blokken. Deze blokken bevatten verschillende gegevens. Deze gegevens worden omgezet door middel van wiskundige functies die ook wel hashfuncties worden genoemd. Door de blokken onderling aan elkaar te koppelen en wiskundige functies te gebruiken bij het verifiëren van de integriteit van de blokken ontstaat er een veilige en volledig gedecentraliseerde manier van dataopslag.\\\\

\item \textbf{P2P}: Tot slot is er ook het concept van Peer-to-peer (courant afgekort als P2P). Voor een gangbare definitie kan gebruik gemaakt worden van de werken van \textcite{Schollmeier2001} en \textcite{IAB5694}. Hierbij wordt gesteld dat P2P bestaat uit verschillende computers (nodes) die onderling met elkaar verbonden zijn. Deze nodes vervullen daarbij de rol van zowel server als client (zogenaamde Servents). Hierdoor kunnen de verschillende nodes diensten en data aan elkaar opvragen en delen zonder een centraal aanspreekpunt (client-server architectuur). Dit is dan ook het achterliggende principe van IPFS. Blockchain is een manier om gegevens~ en gedragsintegriteit (als gedefinieerd in \autocite{Drescher2017}) te waarborgen zonder centrale autoriteit binnen P2P netwerken.\\
\end{itemize}

\noindent Versiebeheer wordt vaak uitbesteed aan derden of zelf gedaan aan de hand van een centrale server. Het nadeel hiervan is dat er één centrale plek is waar het kan mislopen. Stel bijvoorbeeld dat de centrale server gegevens verliest is men alles kwijt. Een ander probleem is dat er binnen versiebeheer een aantal monopolies ontstaan waaronder Microsoft die GitHub kocht in 2018.  Dit staat haaks op de open source beweging die streeft naar een transparante en democratische manier van software ontwikkeling. Door het introduceren van de bovengenoemde technologieën kunnen zowel de bestanden als de nodige informatie voor versiebeheer worden verspreid waardoor er geen centraal punt is en er ook geen commercieel bedrijf bij betrokken is.\\

\noindent De doelstelling van dit onderzoek is om versiebeheer te decentraliseren. Daaronder wordt verstaan overstappen van de klassieke server-client architectuur zoals github (of een lokaal gehost versiebeheer systeem) naar een P2P netwerk. Voor de bestanden wordt gebruik gemaakt van de reeds bestaande IPFS technologie. Hierbij zal een blockchain oplossing worden ontwikkeld om het geheel te ondersteunen (metadata, manifest bestanden,...). -Zie ook \ref{sec:methodologie} Methodologie.-\\

\noindent Om de onderzoeksvraag volledig te beantwoorden kan deze nog verder opgesplitst worden in verschillende deelvragen. Deze vragen komen dan ook chronologisch aan bod om uiteindelijk tot een werkend systeem te bekomen. De verschillende deelvragen die worden behandeld zijn:

\begin{itemize}
\item{Wat zijn de problemen van versiebeheer binnen softwareontwikkeling en hoe worden deze aangepakt door versiebeheer systemen zoals GIT?}
\item{Waarom zouden we van gecentraliseerde (server-client architectuur) versiebeheer systemen overstappen naar een gedecentraliseerde variant?}
\item{Wat zijn de eigenschappen en valkuilen van gedecentraliseerde (P2P) netwerken?}
\item{Hoe gaan protocollen zoals Gnutella en BitTorrent te werk voor Peer-to-peer filesharing?}
\item{Wat is IPFS en hoe vergelijkt het met andere gedecentraliseerde filesharing protocollen?}
\item{Op welke manier biedt IPFS een meerwaarde ten opzichte van klassieke versiebeheer systemen?}
\item{Wat zijn de basisprincipes van Blockchain?}
\item{Hoe kunnen data veilig en integer worden bewaard op een Blockchain netwerk?}
\item{Welke meerwaarde kan blockchain bieden binnen de context van Peer-to-peer netwerken?}
\item{Wat zijn smartcontracts en hoe kunnen ze een meerwaarde bieden binnen blockchain oplossingen? }
\item{Op welke wijze kunnen we IPFS en blockchain combineren tot een werkzaam versiebeheer systeem?}
\end{itemize}

%---------- Stand van zaken ---------------------------------------------------

\section{State-of-the-art}
\label{sec:state-of-the-art}

De manier van werken is gebaseerd op het artikel “Decentralized document version control using ethereum blockchain and IPFS.” \autocite{Nizamuddin2019} In het onderzoek wordt er gebruik gemaakt van smart contracts. Dit is in essentie code die zal uitgevoerd worden als aan bepaalde voorwaarden wordt voldaan. Deze smart contracts worden gebruikt om de verschillende aspecten van versiebeheer en data vast te leggen en uit te voeren. Voor de bestanden binnen het project wordt gekozen voor IPFS om op een gedecentraliseerde manier deze te kunnen opslaan. \\\\
De paper vormt een zeer goede aanzet en ook de werkmethode is uitvoerig beschreven. Toch blijft het zeer abstract. Belangrijke aspecten van versiebeheer worden kort of niet aangehaald waaronder “cloning”, “merging” of “branching”. In de bovengenoemde paper wordt een sterke focus op Ethereum gelegd, ontwikkeling bovenop deze blockchain interpretatie brengt echter significante overhead met zich mee. Zo is de snelheid van het systeem afhankelijk van de capaciteit en belasting van het netwerk op het gegeven moment.\\\\
Deze bachelorproef legt de focus op het ontwikkelen van een concrete toepassing. Ook de meer complexe en technische problemen zullen worden behandeld. De algemene principes van blockchain zullen vrijer worden geïmplementeerd en op een lokaal netwerk van enkele computer worden verspreid. In plaats van een grotere architectuur en implementatie te gebruiken
% Voor literatuurverwijzingen zijn er twee belangrijke commando's:
% \autocite{KEY} => (Auteur, jaartal) Gebruik dit als de naam van de auteur
%   geen onderdeel is van de zin.
% \textcite{KEY} => Auteur (jaartal)  Gebruik dit als de auteursnaam wel een
%   functie heeft in de zin (bv. ``Uit onderzoek door Doll & Hill (1954) bleek
%   ...'')


%---------- Methodologie ------------------------------------------------------
\section{Methodologie}
\label{sec:methodologie}

Er wordt vertrokken vanuit een literatuurstudie om de verschillende elementen van versiebeheer en de reeds bestaande technologieën te verkennen. Vervolgens komen de aspecten van blockchain en IPFS aan bod door middel van een demo waarin wordt gebruik gemaakt van een Word document met verschillende versies.\\\\
Tot slot worden de verschillende aspecten van versiebeheer aan de hand van een demo-applicatie geïllustreerd. Hiervoor zijn er drie hypothetische gebruikers: Alice, Bob en Carol die samen een T-shirt webshop ontwikkelen. Ze zullen hiervoor gebruiken maken van ASP.Net en Visual Studio. Binnen hun ontwikkelingsproces zullen ze een aantal gekende problemen tegenkomen waaronder “merge conflicten” en verschillende “branches”.\\\\
Bij elk van die problemen wordt er gekeken naar hoe Git -een klassiek versiebeheer systeem- dit oplost en hoe er een oplossing kan voorzien worden vanuit de voorgestelde gedistribueerde blockchain benadering. Voor het opstellen van de blockchain wordt gebruik gemaakt van C\# en Nethereum \autocite{Nethereum}. Aangezien er wordt gesteund op de IPFS API wordt er gebruik gemaakt van de open source bibliotheek net-ipfs-client-http geschreven door \textcite{IPFSClient}. De bedoeling is om op het einde van de bachelorproef tot een werkend prototype te komen dat gebruikt kan worden voor verschillende doeleinden.

%---------- Verwachte resultaten ----------------------------------------------
\section{Verwachte resultaten}
\label{sec:verwachte_resultaten}

Het eindresultaat van de Bachelorproef is om op een onderbouwde manier een prototype aan te reiken om op gedecentraliseerde wijze aan versie beheer te gaan doen. De voorgestelde werkwijze wordt grondig vergeleken met Git op de volgende twee punten:\\\\

\begin{itemize}
\item Snelheid van een transactie: hoe lang duurt het om bewerkingen zoals pull requests en branching toe te passen op een project en/of branch?
\item Performantie qua geheugengebruik: hoe efficiënt wordt er binnen de algoritmen van de oplossing omgesprongen met geheugengebruik? Hiermee wordt zowel het extern geheugen (Hardeschijf, SSD) als het werkgeheugen bedoelt.\\\\
\end{itemize}

\noindent De verwachting is dat de implementatiesnelheid lager zal zijn dan met de klassieke Git-systemen, omdat blockchain van nature vrij omslachtig en intensief is. De performantie van het geheugensysteem zal eveneens slechter scoren ten opzichte van Git. De blockchain implementatie waarborgt echter een hogere mate van data integriteit .  

%---------- Verwachte conclusies ----------------------------------------------
\section{Verwachte conclusies}
\label{sec:verwachte_conclusies}

Gedecentraliseerd versiebeheer door middel van Blockchain en IPFS is zeker technisch mogelijk. De voordelen die het biedt zijn niet alleen zuiver ideologisch.De inherente garantie op data integriteit en het weghalen van een centraal ``\textit{point of failure}'' is interessant voor grote bedrijven met een groot aantal aan verschillende projecten. Er zijn echter een aantal nadelen verbonden waaronder de omslachtige procedure en het intensief gebruik van computertechnische middelen. Dit maakt het voor kleine bedrijven minder interessant.